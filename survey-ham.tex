\documentclass[12pt]{article}

\usepackage{sbc-template}
\usepackage{graphicx,url}

\usepackage[brazil]{babel}   
\usepackage[latin1]{inputenc}
\usepackage[T1]{fontenc} % inserir as url 

\usepackage[usenames,dvipsnames]{xcolor}

\usepackage{tikz}
\usepackage{tkz-kiviat}
\usetikzlibrary{positioning}
\usetikzlibrary{arrows}
\usepackage{hyperref}

\hypersetup{
    hidelinks, % Remove colora��o e caixas
    unicode=true,   %Permite acentua��o no bookmark
    linktoc=all %Habilita link no nome e p�gina do sum�rio
}

%\usepackage[round,authoryear]{natbib}

\usepackage{xargs}

\usepackage[colorinlistoftodos,prependcaption,textsize=tiny]{todonotes}
\newcommandx{\needcite}[2][1=]{\todo[linecolor=red,backgroundcolor=red!25,bordercolor=red,#1]{#2}}
\newcommandx{\warning}[2][1=]{\todo[linecolor=yellow,backgroundcolor=yellow!25,bordercolor=yellow,#1]{#2}}
\newcommand\todoin[2][]{\todo[inline, caption={~}, #1]{
\begin{minipage}{\textwidth-4pt}#2\end{minipage}}}

%%%WORKAROUND TO MAKE HYPERREF COMPATIBLE WITH SBC BIBLIOGRAFY STYLE. Author: Douglas P. Pasqualin
\makeatletter
  \def\@lbibitem[#1]#2{%
    \@skiphyperreftrue
    \H@item[%
      \ifx\Hy@raisedlink\@empty
        \hyper@anchorstart{cite.#2\@extra@b@citeb}%
        \hyper@anchorend
      \else
        \Hy@raisedlink{%
          \hyper@anchorstart{cite.#2\@extra@b@citeb}\hyper@anchorend
        }%
      \fi
      \hfill
    ]%
    \@skiphyperreffalse
    \if@filesw
      \begingroup
        \let\protect\noexpand
        \immediate\write\@auxout{%
          \string\bibcite{#2}{#1}%
        }%
      \endgroup
    \fi
    \ignorespaces
  } 
\makeatother  
%%%END OF WORKAROUND. DO NOT CHANGE!

\sloppy

\title{Um estudo sobre o uso de mem�rias STT-RAM vol�teis em diferentes n�veis de mem�ria cache}

\author{Giovane de O. Torres\inst{1}, Rodrigo M. Duarte\inst{1}}

\address{Centro de Desenvolvimento Tecnol�gico -- UFPel \\
  96.010-610 -- Pelotas -- RS -- Brasil
  \email{\{gdotorres,rmduarte\}@inf.ufpel.edu.br}
}

\begin{document} 

\maketitle
   
\begin{resumo}
	Atualmente existe a necessidade de melhorias na �rea de mem�rias, sendo o principal objetivo reduzir lat�ncia e consumo energ�tico. Este trabalho faz uma avalia��o de STT-RAM vol�teis, uma tecnologia emergente de mem�ria que pode suprir esta demanda. Por�m, sua aplica��o necessita de projetos elaborados para manter o correto armazenamento dos dados.
  
\end{resumo}

\section{Introdu��o}

Com as diversas evolu��es que ocorrem na �rea da computa��o, s�o sempre interessantes e importantes poss�veis descobertas e inova��es tecnol�gicas as quais podem alavancar melhorias em dispositivos eletr�nicos. Estas melhorias podem incluir redu��es tanto na lat�ncia como no consumo energ�tico. Atualmente, uma das �reas que demandam estes aprimoramentos s�o as mem�rias -- estas apresentam grande desafio na escalabilidade~\cite{poremba:12,young:15} e desperd�cio de energia~\cite{wang:13,Li2015}. Al�m disso, a mem�ria � considerada como um dos componentes cr�ticos em computadores~\cite{perez:10,zou:15}. Portanto, buscar alternativas que consigam extrair mais desempenho de mem�rias � essencial para o futuro.

Dentro deste contexto, tecnologias de mem�ria emergentes visam melhorar o desempenho deste componente eletr�nico. As MNVs (mem�rias n�o vol�teis) est�o inseridas dentro das novas tecnologias, sendo promissoras em diversos aspectos~\cite{meena:14,young:15,zhao:15}, os quais incluem:

\begin{itemize}
	\item Prover baixo consumo energ�tico;
	\item Melhor escalabilidade;
	\item Maior densidade por c�lula de mem�ria;
	\item Garantia de armazenamento de informa��es na c�lula de mem�ria sem a necessidade de efetuar opera��es de \textit{refresh}.
\end{itemize}

Apesar dos diversos pontos positivos que as MNVs apresentam, as mesmas n�o s�o largamente utilizadas por apresentarem alguns desafios que devem ser superados. Um dos problemas que esta categoria de mem�rias apresenta est� na durabilidade do material que comp�em suas c�lulas, o qual � relativamente pequeno se comparados com tecnologias de mem�rias atuais, e.g. DRAM e SRAM~\cite{mittal:15}. Outra dificuldade encontrada em MNVs ocorre quando deve-se fazer uma opera��o de escrita -- tanto a lat�ncia como o consumo de energia nestas opera��es s�o considerados altos~\cite{mittal:16}.

Dentro do contexto de MNVs, existe diversos tipos de mem�rias as quais s�o diferenciadas pelos materiais que comp�em suas c�lulas. Mem�rias estudadas dentro da bibliografia incluem: PCRAM (\textit{Phase Change Random Access Memory}), STT-RAM (\textit{Spin Transfer Torque Random Access Memory}), RRAM (\textit{Resistive Random Access Memory}), MRAM (\textit{Magnetic Random Access Memory}), FERAM (\textit{Ferroelectric RAM}), al�m de outras mem�rias mais emergentes como \textit{Racetrack Memory} e mem�ria molecular~\cite{meena:14}. Dentro deste amplo contexto, PCRAM, STT-RAM e RRAM s�o consideradas como NVMs que melhor podem compor sistemas de mem�ria com grande efici�ncia e alta densidade, devido as propriedades melhoradas de escalabilidade e n�o volatilidade~\cite{young:15}, al�m de prometerem ser empregadas como mem�rias universais devido � capacidade de armazenamento e lat�ncias compar�veis a tecnologias como a DRAM~\cite{mittal:16}.

Para a melhor extra��o de desempenho de MNVs, existe diversas frentes de pesquisas que incluem modifica��es em arquiteturas bem como t�cnicas que possibilitem efetuar melhoras nas mem�rias. Uma das possibilidades incluem a redu��o da capacidade de reten��o das MNVs com a finalidade de permitir que haja menor lat�ncia nas opera��es de mem�ria, especialmente nas escritas. Esta t�cnica � chamada de \textbf{relaxamento de reten��o}~\warning{Talvez mudar este nome}, e pode ser aplicada ao utilizar-se das MNVs em \textit{caches} ou na mem�ria principal -- isto porque estas mem�rias est�o sujeitas a consider�veis quantidades de escritas, sendo os dados alterados com o passar do tempo. Com isto, reter informa��es por longos per�odos de tempo sem modifica��es torna-se dispens�vel. A utiliza��o de relaxamento de reten��o � vista como uma t�cnica potencial para difundir o uso de MNVs em sistemas de mem�ria. Dentro do melhor do conhecimento dos autores, o relaxamento de reten��o teve at� o presente momento seus efeitos testados na STT-RAM, sendo objeto de estudo de diversos trabalhos. No trabalho de~\cite{mittal:16}, a combina��o de STT-RAM com relaxamento de reten��o � chamada de \textbf{STT-RAM vol�til}.

Com isto, este trabalho tem como objetivo efetuar um estudo sobre as formas de utiliza��o da STT-RAM vol�til nos diferentes n�veis dentro de uma hierarquia de mem�ria -- visando quais as melhores abordagens para o uso desta mem�ria, bem como estudar a viabilidade da utiliza��o de STT-RAM vol�til.

\warning[inline]{Inserir um par�grafo sobre a estrutura a seguir do texto? Se��o 2 apresenta bla bla bla..}
\warning[inline]{Incluir algo sobre as discuss�es/conclus�es quando estas se��es estiverem prontas}

\section{O que s�o STT-RAM Vol�teis}
	Aqui ser� explicado como � poss�vel quebrar a caracter�stica de n�o-volatilidade das STT-RAM.
	
	Quais s�o os efeitos da quebra da n�o-volatilidade.	
      
\section{Trabalhos Relacionados}

\todoin[size=\scriptsize,linecolor=gray,backgroundcolor=gray!25,bordercolor=gray]{
\begin{center}
\begin{tabular}{lcp{7cm}}
	\multicolumn{1}{c}{\textbf{Artigo}} & \multicolumn{1}{c}{\textbf{N�veis de Cache}} & \multicolumn{1}{c}{\textbf{Descri��o}} \\
	\cite{Smullen2011} & L1,L2,L3 & Primeiro artigo a abordar STT-RAM vol�til \\
	\cite{li2011} & L1 e L2 & \\
	\cite{Sun2011} & L1, L2 e L3 & \\
	\cite{Jog2012} & L2 & \\
	\cite{Li2013} & L2 & \\
	\cite{Chang2013} & L3 & \\
	\cite{Li2015} & L1 & \\
	\cite{Kim2015} & & \\
	\cite{qiu2016} & & \\
	\cite{kim2016} & & \\
\end{tabular}
\end{center}
}

% Smullen2011 %

	O primeiro trabalho que faz um estudo e apresenta o conceito de STT-RAM vol�til � visto em~\cite{Smullen2011}. Este trabalho primeiramente explica como a STT-RAM pode ser vol�til -- efeito atingido atrav�s da redu��o da �rea planar da camada livre da MTJ da c�lula de mem�ria. O artigo apresenta um modelo de STT-RAM vol�til, exibe t�cnicas para otimiza��o desta mem�ria para prover melhorias no desempenho da escrita. Este modelo � testado em todos os n�veis de \textit{cache} nas quest�es de desempenho, energia e \textit{energy-delay}. Por fim, com a perda de reten��o da STT-RAM, os dados em mem�ria n�o podem ser perdidos. Com isto, o trabalho estuda um esquema simples de \textit{refresh} que adiciona pouco \textit{overhead} sobre o sistema de mem�ria.

	A primeira avalia��o � efetuada sobre a rela��o entre as velocidades de leitura e escrita de uma c�lula STT-RAM. Chega-se a conclus�o de que a fim de reduzir a lat�ncia das escritas, deve-se aumentar o tempo das leituras sem que isto deteriore o desempenho geral das leituras. Fazendo-se o processo reverso, i.e., aumentando o tempo de escrita numa c�lula da STT-RAM, a lat�ncia de leitura tende a diminuir. Outra an�lise � feita diminuindo a �rea planar da c�lula STT-RAM, sendo esta diretamente proporcional ao tempo de reten��o do dado (Uma c�lula de $10F^2$ tem tempo de reten��o de somente $56\mu s$). Ao reduzir o tempo de reten��o, melhorias s�o percebidas na lat�ncia de escritas, bem como na energia em opera��es de leitura e escrita.

	O trabalho faz ent�o avalia��es do impacto do uso da STT-RAM vol�til atrav�s de simula��es com \textit{benchmarks}. Primeiramente, efetua-se a troca de todas as \textit{caches} SRAM por STT-RAM vol�teis, verificando que em quest�es de desempenho a maioria dos casos estudados existem perdas devido ao acr�scimo nos tempos de escrita em mem�ria. Em contrapartida, h� uma redu��o de mais de 3x na fuga de energia ao usar STT-RAM. Outras avalia��es que geraram resultados tamb�m foram realizadas, por�m considerando hierarquias de mem�rias h�bridas, podendo incluir tanto SRAM como os diferentes modelos de STT-RAM mostrados anteriormente. Os modelos de \textit{cache} h�bridos com STT-RAM de �rea $19F^2$ e $10F^2$ conseguem ter desempenho igual ou melhor comparado com \textit{caches} SRAM. Nestes modelos h�bridos, a redu��o do consumo energ�tico � menor se comparada com \textit{caches} puramente de tecnologia STT-RAM, por�m ainda assim h� ganhos consider�veis comparando com as SRAM.

	Por fim, considerando que a mem�ria estudada passa a ser vol�til, o artigo prop�e e utiliza-se de uma pol�tica de \textit{refresh} simples, tal qual a DRAM, iterando por cada linha para aplicar um \textit{refresh}. A �ltima an�lise do trabalho envolve algumas configura��es de hierarquias de \textit{caches} as quais s�o testadas com este esquema de \textit{refresh}, sendo comparadas com as hierarquias j� visas anteriormente, bem como o \textit{baseline} (\textit {caches} SRAM). Conclui-se que ao utilizar \textit{designs} de \textit{caches} STT-RAM de $19F^2$ com \textit{refresh} conseguem superar tanto no consumo quanto no desempenho os projetos de $32F^2$, al�m de prover melhor desempenho. 

% Li2011 %

	No trabalho apresentado em \cite{li2011}, � realizado um \textit{tradeoff} sobre o desempenho, confiabilidade e consumo energ�tico das STT-RAM, aplicados aos requisitos no n�vel de arquitetura. No trabalho � apresentada a modelagem da STT-RAM para a redu��o do tempo de reten��o dos dados. O objetivo � a aplica��o em n�veis mais espec�ficos da hierarquia de mem�ria, com foco em \textit{caches on-chip}.
	
	Primeiramente � feita uma an�lise para verificar quais s�o os impactos na quebra da n�o volatilidade da STT-RAM, para isso � realizada uma simula��o de tr�s diferentes desenhos do MTJ com $45$x$95nm$ de tamanho. S�o realizadas duas otimiza��es da \textit{baseline}, chamadas de Opt1 e Opt2. Sendo a \textit{baseline} com seu tempo de reten��o de dados de 4.27 anos e os outros dois otimizados para chaveamento (quando menor o tempo de reten��o dos dados, menor o tempo necess�rio para chaveamento). Nos testes faz-se perceber que ao reduzir o tempo de reten��o dos dados de 4 anos para $265 \mu s$, a corrente decai de $164.5 \mu A$ para $71.4 \mu A$ para um tempo de chaveamento de $10ns$ e temperatura de $350K$. Ao se colocar uma corrente de $125 \mu A$ o tempo de chaveamento dos tr�s MJTs varia de $29.8ns$ para $3.6ns$, um ganho de ate 8 vezes.
	
	� demostrando tamb�m a depend�ncia da temperatura. A estabilidade da barreira magn�tica do MJT � sens�vel a temperatura de trabalho, assim, ao se aumentar a temperatura de $275K$ para $350K$ a corrente de chaveamento tende a diminuir. Logo apos, o trabalho apresenta estat�sticas de padr�o de acesso a \textit{cache}. Uma simula��o � feita em um processador \textit{quad-core} sobre as seguintes configura��es de \textit{cache} L1 de dados e L2:

\begin{table}[]
\centering
\begin{tabular}{lcc}
\hline
                             & \multicolumn{1}{l}{\textit{Cache} de dados L1} & \multicolumn{1}{l}{\textit{Cache} L2} \\ \hline
Tamanho (Bytes)              & 32768                                 & 4194304                       \\ \hline
Associatividade              & 8                                     & 16                            \\ \hline
Tamanho do bloco (bytes)     & 64                                    & 64                            \\ \hline
Lat�ncia de Leitura (ciclos) & 3                                     & 14                            \\ \hline
\end{tabular}
\caption{Configura��o das \textit{caches} L1 de dados e L2, para a simula��o.}
\label{my-label}
\end{table}	
	
	 O teste � realizado usando quatro aplica��es do SPEC \textit{benchmark} que s�o: 401.bzip2, 433.milc, 434zeusmp e 470lbm. Cada uma destas aplica��es � executada em um \textit{core} diferente, sendo simulado um total de 2 bilh�es de instru��es. Para a \textit{cache} L1 mais de 95\% dos dados s�o acessados nos primeiros $10^{5}$ ciclos de \textit{clock}, apos isto, estes dados s�o carregados ou atualizados. Em alguns casos, como no \textit{benchmark} 401.bzip2 o n�mero pode alcan�ar at� 99\% e que comportamentos similares se apresentaram na \textit{cache} L2. Estas observa��es demostram que uma pequena por��o da \textit{cache} de dados ser� ativa por longos per�odos, como exemplo do \textit{benchmark} 401.bzip2 que o tempo entre uma escrita e o �ltimo tempo em que os dados s�o lidos, excede $10^6$ ciclos de \textit{clock}. Este tempo entre leitura e escrita � utilizado para determinar o tempo m�nimo de reten��o de dados pelo MTJ.
	 
	 No final � apresentado um modelo de \textit{cache} h�brida conjunto associativa com $n$-vias, onde a mesma � dividida em dois modelos de MTJ, um com baixa reten��o de dados (utilizando os dados apresentados anteriormente) e outra com alta reten��o. Cada bloco da \textit{cache} otimizada cont�m um contador que verifica quanto tempo os dados est�o armazenados no bloco. Caso estes estejam por muito tempo, o sistema copia os dados da \textit{cache} otimizada para a n�o otimizada. Para avaliar o desempenho, este novo modelo � comparado a um usando STT-RAM sem otimiza��o. Os resultados mostram que usando metade das vias de uma \textit{cache} L2 com 16 vias otimizada, o desempenho do novo modelo chega a ser 80\% melhor. A conclus�o � que a redu��o da n�o volatilidade da STT-RAM e o uso de uma \textit{cache} h�brida, podem aumentar o desempenho e reduzir o consumo energ�tico. Estas conclus�es s�o retiradas da an�lise do modelo proposto e os testes realizados com os \textit{benchmarks}.

% Sun2011 %
	
	Em \cite{Sun2011} tamb�m � apresentada a proposta de implementa��o de \textit{caches} h�bridas, por�m � apresentada a preocupa��o com a correta reten��o dos dados na STT-RAM. Para se conseguir lat�ncias de escrita suficientemente baixas, � necess�rio reduzir muito o per�odo de reten��o dos dados, sendo que dependendo do tempo que o dado vai permanecer na \textit{cache}, um processo de \textit{refresh} dos dados acaba se tornando necess�rio. Assim � proposta a implementa��o de um sistema de \textit{refresh} din�mico. Para cada bloco da \textit{cache} � inserido um contador que monitora o tempo de reten��o dos dados, se nenhuma escrita/leitura ocorre em um determinado per�odo limite pr� definido, um \textit{refresh} ocorre. A ideia do uso do contador � para evitar a necessidade de \textit{refreshes} desnecess�rios.
	
	Uma \textit{cache} com dois modelos de MTJ � proposto, um com baixa e outro com alta reten��o. Os dados s�o movidos das regi�es de alta para as de baixa reten��o, ou vice versa, atrav�s das seguintes politicas: 
		\begin{itemize}
			\item \textit{Write Intensive}: se a intensidade de escrita � alta, e os dados se encontram em uma regi�o de alta reten��o, os dados s�o migrados para uma regi�o de baixa.
			\item \textit{Read intensive}: se os dados se encontram em uma regi�o de baixa reten��o, ent�o estes s�o migrados para uma regi�o de alta.
			\item \textit{Neither write nor read intensive}: Os dados s�o mantidos em uma regi�o de alta reten��o ou migrados para mem�ria principal.   
		\end{itemize}
	
	A principal ideia para a migra��o de regi�es de mem�ria esta ligada tamb�m a redu��o de \textit{refreshes}, j� que o modelo de \textit{refresh} utilizado no modelo proposto � o mesmo utilizado em mem�rias DRAM.
	
	Para os testes, s�o simulados tr�s modelos de otimiza��o de MTJ, um com baixa (lo), m�dia (md) e alta reten��o (hi). em uma arquitetura \textit{quad-core} com tr�s n�veis de \textit{cache} L1, L2 e L3, ambas usando STT-RAM. Os resultados dos testes mostram que a melhor configura��o para uma hierarquia de dois n�veis (L1 e L2) � usando L1-lo e uma L2 hibrida com L2-lo para regi�o de baixa reten��o e L2-md para a regi�o de alta. J� para uma \textit{cache} de 3 n�veis a melhor configura��o se deu usando L1-lo, L2-lo com L2-md e una L3 com a configura��o de uma L1-lo para a regi�o de baixa reten��o e L3-md para regi�o de alta.
	
	Os resultados dos testes mostram que o uso das \textit{caches} h�bridas propostas utilizando os m�todos de \textit{refresh} apresentados, tem um ganho de desempenho de at� 99.8\% se comparada ao uso de um SRAM convencional. J� no quesito consumo de energia, o modelo apresentado tem ganhos de at� 70\% se comparado a outros trabalhos e.g~\cite{Smullen2011}. O trabalho conclui que o uso de mem�rias h�bridas (baixa e alta reten��o) apresentam ganhos n�o somente em desempenho como tamb�m em consumo de energia. O uso de m�ltiplos n�veis de reten��o podem alcan�ar uma redu��o de at� 73.8\% de redu��o de energia se comparado a implementa��es que mesclam SRAM/STT-RAM e se comparado a outros trabalhos anteriores que apresentam o uso de STT-RAM, consegue um aumento de desempenho de at� 5.5\% e uma redu��o energ�tica de at� 30\%.

% Jog2012 %

	O trabalho exibido em~\cite{Jog2012} tem o foco de ajustar o tempo de reten��o de dado em mem�ria de acordo com o tempo necess�rio para execu��o do \textit{refresh} da \textit{cache} de �ltimo n�vel (\textit{Last Level Cache} -- LLC), com a finalidade de obter ganhos no desempenho e energia. S�o propostos e estudados tr�s modelos de STT-RAM: (i) um sem reduzir a reten��o, (ii) outro o qual tem seu tempo de reten��o reduzido para $1s$, considerado grande o bastante para o tempo de escrita da maioria das linhas de \textit{cache}, n�o acarretando em \textit{overhead}. E por fim, (iii) um modelo com tempo de reten��o de $10ms$ o qual � necess�rio utilizar uma t�cnica de \textit{refresh}, embora a lat�ncia e consumo de energia sejam menores para efetuar uma escrita neste caso.

	A STT-RAM vol�til neste caso funciona de maneira simplificada: Todos os blocos os quais ap�s um determinado tempo est�o por perder o dado, efetua-se um \textit{write-back} nos mesmos, utilizando-se de um contador de poucos \textit{bits}. Esta abordagem acaba prejudicando o desempenho por que ao final do tempo de reten��o, existir�o diversos \textit{write-backs}, causando um grande \textit{overhead}. Al�m disto, se um bloco o qual � constantemente lido perder a informa��o, haver� maior quantidade de \textit{misses} de leitura na \textit{cache}. Para contornar principalmente o primeiro problema, � proposta a t�cnica de \textbf{\textit{Cache Revive}}, o qual adiciona um pequeno \textit{buffer} que amortiza as custosas escritas que seriam feitas diretamente na STT-RAM.

	Os resultados deste trabalho procuram avaliar, dentre os tr�s modelos, algumas configura��es de estudo de cache, as quais foram: S-1 (SRAM de $1MB$); S-4 (Caso hipot�tico onde a SRAM tem $4MB$, por�m apresenta a mesma lat�ncia de opera��o de S-1); M-4 (STT-RAM n�o vol�til); V-M-4($1s$) (STT-RAM vol�til com tempo de reten��o de $1s$); V-M-4($10ms$) (STT-RAM vol�til com tempo de reten��o de $10ms$) e R-M-4($10ms$) (STT-RAM vol�til utilizando a t�cnica de \textit{cache revive}). 

	Em quest�es de desempenho, em m�dia o desempenho de R-M-4 ($10ms$) s� perde para o modelo hipot�tico S-4, tendo \textit{speedup} de quase 20\% ao simular os \textit{benchmarks} do PARSEC se comparados com o \textit{baseline} S-1. Analisando o consumo energ�tico, percebe-se um ganho de 44\% na configura��o M-4 para a S-1, o que � esperado dado que a STT-RAM gera pouca fuga de energia. Ao tornar a STT-RAM vol�til, tem-se maior fuga de energia, ainda que inferior se comparada a SRAM. Na m�dia, h� ganhos em energia de 11\% no caso de estudo R-M-4 ($10ms$) comparado com V-M-4($1s$), al�m de melhoria em 18\% na energia em rela��o � configura��o M-4 (\textit{baseline} da STT-RAM), concluindo assim que energeticamente o uso de STT-RAM vol�til com o esquema de \textit{cache revive} atingiu os melhores resultados na m�dia.

% Li2013 %

	Em~\cite{Li2013}, prop�e-se uma t�cnica chamada de CCear (\textit{Cache Coherence Enabled Adaptative Refresh}) a qual visa reduzir a quantidade de \textit{refreshes} em uma STT-RAM vol�til. Este trabalho tem como diferencial obsevar a informa��o de coer�ncia dos blocos de \textit{cache} L2 com a finalidade de atenuar o \textit{overhead} gerado por opera��es de \textit{refresh} na mem�ria. Dentro da t�cnica de CCear, em cada bloco que � compartilhado s�o feitas $n$ opera��es de \textit{refreshes} ap�s o carregamento da mem�ria principal ou depois de um \textit{write-back} da \textit{cache} L1.

	Utilizando-se do CCear, � feito um comparativo do uso de uma arquitetura com suporte a este mecanismo com a pol�tica de DRAM \textit{refresh} proposta em~\cite{Smullen2011}, sendo analisadas as quest�es energ�ticas e o IPC (Instru��es por ciclo). Al�m destes dois casos, o trabalho ainda exibe resultados do que seria a situa��o ideal para a implementa��o de STT-RAM vol�teis. Na energia, o modelo de DRAM \textit{refresh} exibe consumo alto se comparado ao caso ideal, enquanto que ao usar o CCear h� uma redu��o em rela��o � pol�tica de DRAM \textit{refresh} na m�dia em cerca de 10\%. Avaliando o IPC, utilizando CCear existe uma melhora no IPC comparado ao DRAM \textit{refresh} entre cerca de 3\% a 7\%, ocorrido devido � menor quantidade de conflitos entre \textit{refreshes} e leituras na \textit{cache}.

	O artigo conclui dizendo que o \textit{overhead} para armazenamento de informa��es necess�rias para o funcionamento do CCear � desprez�vel - Inicialmente, � de menos de 0,2\%, visto que cada bloco da LLC precisa de um bit para indicar se est� expirado ou n�o. Al�m disto, cada bloco da LLC tamb�m usa um contador de 4 \textit{bits} para controlar quantos \textit{refreshes} s�o feitos. Tamb�m conclui que o CCear pode adaptativamente minimizar a quantidade de opera��es de \textit{refresh} necess�rias para o uso de STT-RAM vol�til.

% Chang2013 %

	J� o trabalho apresentado por \cite{Chang2013}, � realizada uma compara��o entre tr�s diferentes tipos de mem�rias aplicadas ao �ltimo n�vel de \textit{cache}, s�o elas a SRAM, STT-RAM e eDRAM. A ideia � otimizar a SRAM para baixa corrente de fuga, a STT-RAM para baixo consumo energ�tico de escrita e a eDRAM usando a ideia de \textit{dead-line prediction} para reduzir a quantidade de \textit{refreshes} desnecess�rios. A implementa��o proposta de STT-RAM neste trabalho � um MTJ com reten��o de apenas 1 segundo, n�o s�o experimentados outros tempos de reten��o de dados devido a necessidade de \textit{buffers} e unidades de \textit{refresh} para manter os dados corretamente armazenados. Os autores n�o se preocupam com isso porque o foco principal do artigo � o uso de eDRAMs.
	
	Para os testes � simulada uma arquitetura com 8 \textit{cores} e tr�s n�veis de \textit{cache} onde, L1 e L2 s�o \textit{caches} usando SDRAM de alto desempenho e L3 � uma \textit{cache} de 32Mb, este �ltimo n�vel de \textit{cache} com os tr�s diferentes tipos de mem�ria (SDRAM, STT-RAM e eDRAM). Os resultados mostram que o uso de STT-RAM consome at� 48\% menos energia que o uso de SRAM, contudo se a quantidade de escritas na mem�ria � alta, STT-RAM consome maior quantidade de energia. Para cargas de trabalho com escrita intensiva a eDRAM se mostrou melhor na quest�o de consumo energ�tico, ficando 36\% mais eficiente que a SRAM de baixo consumo e 17\% comparado a STT-RAM. No quesito desempenho, em m�dia SDRAM apresenta os melhores tempos, porem em cargas de trabalhos dominadas por leituras, STT-RAM apresentam melhores resultados que as demais. 
	
	Outra observa��o feita no trabalho s�o os impactos que o tamanho e tecnologia exercem sobre as mem�rias. Caches que possuem alta densidade de mem�ria usando STT-RAM tendem a apresentar menor consumo de energia. Isso se d� pelo fato de que quanto maior a \textit{cache} em m�dia, menor ser� o n�mero de \textit{cache miss}, sendo assim menor o n�mero de atualiza��es. No quesito tecnologia, quanto menor a c�lula STT-RAM, menor sera o consumo de energia, pois o tempo de escrita ser� menor. STT-RAM com tecnologia de $22nm$ s�o mais eficientes energeticamente do que as de $45nm$, por�m o tempo de reten��o pode sofrer perda de estabilidade devido a instabilidade t�rmica.
	
	Como conclus�o os autores demostram que o uso de eDRAM com a t�cnica de \textit{dead-line prediction} apresentou os melhores resultados, por�m a necessidade de \textit{caches} de desempenho elevado e alta densidade ser�o necess�rias no futuro. Assim o uso de novas tecnologias como a eDRAM e a STT-RAM s�o promissoras.
	
% Li2015 %

	\newcommand{\nr}{$N$-\textit{refresh}}
	\newcommand{\nrdl}{$N$-\textit{refresh}-DL}

	No trabalho de \cite{Li2015}, tamb�m � proposta uma \textit{cache} usando STT-RAM com baixo tempo de reten��o de dados e com a necessidade de \textit{refresh}. Por�m o trabalho vai al�m, neste os autores se preocupam com a ordena��o dos dados na \textit{cache} para reduzir o n�mero de \textit{refreshes}. No trabalho � proposto uma ordena��o dos dados em tempo de compila��o para extrair os benef�cios da STT-RAM e ao mesmo tempo reduzir o consumo de energia atrav�s da redu��o da quantidade de \textit{refreshes} na mem�ria. O trabalho prop�e uma nova metodologia de \textit{refresh} para mem�ria chamado de~\nr. Os dados que s�o escritos mas n�o s�o utilizados por um grande per�odo de tempo ou sofrem uma reescrita, apos um determinado tempo limite, s�o migrados para a mem�ria principal, descartando assim a necessidade de \textit{refresh} da \textit{cache}. Tamb�m s�o propostos dois modelos de organiza��o de dado em mem�ria no c�digo compilado, s�o eles: ILP - \textit{Integer Linear Programming}(ILP) e \textit{Heuristic Data Layout}(DL), ambos utilizam t�cnicas sub-�timas para ordenar os dados entre os blocos da \textit{cache}.
	
	Para os experimentos os autores simulam uma arquitetura de um processador \textit{single-core} de 500Mhz e com uma \textit{cache} de 16Mb, com blocos de 32b com 4 vias. Tempo de reten��o de dados de $26.5\mu s$. O metodo de \textit{refresh} e as t�cnicas de otimiza��o em tempo de compila��o s�o empregadas em outros dois metodos de \textit{refresh} presentes na literatura~\cite{Sun2011},~\cite{Jog2012}. 
	
	Os resultados demostram que o uso da t�cnica~\nrdl, reduziu em apenas 2.0\% a quantidade de \textit{refreshes} se comparado a mesma t�cnica sem o uso de heur�stica (somente usando~\nr). Por�m os outros m�todos j� conseguiram melhores resultados se comparado a outros modelos da literatura, alcan�ando melhorias de at� 38.0\%.
	
	Tamb�m � feito a analise da t�cnica apresentada a diferentes caracter�sticas da \textit{cache} como tamanho, tamanho do bloco, lat�ncia de escrita, tempo de reten��o e o mudan�as no tempo de \textit{refresh}. Para tamanhos de blocos pequenos, as t�cnicas apresentadas fornecem melhores desempenhos por explorar melhor a localidade dos dados, por�m para \textit{caches} com blocos pequenos a perda de desempenho devido a baixa taxa de \textit{cache hit}, j� para blocos grandes as t�cnicas n�o apresentam ganhos.
	
	Em rela��o ao tamanho da \textit{cache}, quanto maior a \textit{cache}, maior se torna a quantidade de \textit{refresh} necess�rios contudo, o m�todo de organiza��o dos dados, aproveita melhor o espa�o da \textit{cache}. Assim, aumentar o tamanho da \textit{cache}, n�o surte efeito no desempenho, pois a nova �rea de mem�ria quase n�o � utilizada. J� em rela��o ao aumento da quantidade de \textit{refresh}, a taxa de \textit{cache hit} aumenta pois menos blocos se tornam invalidados em decorr�ncia da baixa reten��o.
	
	Os dois �ltimos itens (lat�ncia de escrita e reten��o), quanto maior a lat�ncia de escrita maior � o tempo de execu��o do c�digo e maior se torna a quantidade de \textit{refresh} necess�rios, assim baixas lat�ncias de escrita apresentam melhores rendimentos. Quanto a reten��o, quanto maior a reten��o dos dados, menor s�o as taxas de \textit{refresh} necess�rias, por�m a uma degrada��o no uso das t�cnicas de localidade de dados, pois a menos espa�o para a otimiza��o se torna limitado.
	
	Como conclus�o, o trabalho exp�em que os m�todos propostos apresentam melhor rendimento energ�tico, por explorar melhor a localidade dos dados e diminuir a quantidade de \textit{refresh} necess�rio e mant�m um desempenho equivalente as t�cnicas apresentadas em outros trabalhos. 

% Qiu2016 %	
	
	No trabalho de~\cite{qiu2016}, � proposto um m�todo de escalonamento de \textit{loops} para melhorar a localidade dos dados na mem�ria e assim reduzir a quantidade de \textit{refresh}. O trabalho prop�em que os dados sejam acessados em uma determinada ordem, fazendo com que estes sejam "realimentados" por uma leitura e ou uma escrita. Isto se d� pelo fato de que, se o tempo de escrita/leitura � menor que o tempo de reten��o dos dados na STT-RAM, o \textit{refresh} se torna dispens�vel. Os autores defendem o fato de que alinhando os dados em mem�ria em uma determinada forma, reduz-se a quantidade de energia consumida e aumenta-se o desempenho.
	
	Os testes realizados foram a simula��o de microcontrolador acessando uma mem�ria STT-RAM externa. Foram avaliadas tr�s frequ�ncias de opera��o diferentes (100, 200 e 500MHz) e escalas de \textit{loops} de 100x100(x100), 200x200(x200), 500x500(x500) e 1000x1000(x1000). Em m�dia a t�cnica de escalonamento de \textit{loops} os ciclos de acesso a mem�ria (incluindo \textit{refresh}) s�o reduzidos em 54.6\%(100x100),68.7\%(200x200),82.7\%(500x500) e 89.1\%(1000x1000), em todas as frequencias testadas.
	
	J� no consumo de energia din�mica, as t�cnicas reduziram a energia consumida em 85.0\% para 100Mhz, 75.8\% em 200Mhz e 59.5\% em 500Mhz. A explica��o para esta redu��o de energia est� no fato de que, quanto menor a frequ�ncia, maior se torna o tempo de escrita em mem�ria, assim o espa�o de otimiza��o do \textit{loop} � maior, alcan�ando melhores resultados. Outro item � o tamanho dos \textit{loops}, quanto maior a escala do \textit{loop} (1000x1000) maior se torna a distancia entre uma leitura ap�s uma escrita, assim a t�cnica de reescalonamento de \textit{loops} apresenta melhores resultados, por reordenar estes acessos.
	
	Uma outra observa��o � feita no trabalho. O quanto o tempo de reten��o dos dados ajuda a reduzir o consumo de energia e aumenta o desempenho. Observa-se que quanto maior o tempo de reten��o, o escalonamento de \textit{loops} perde desempenho. Isso se d� pelo fato de que, o n�mero de itera��es do \textit{loop} comparado ao tempo de reten��o � muito menor, assim, mais atualiza��es se tornam necess�rias na mem�ria, prejudicando assim o uso da t�cnica.
	
	Como conclus�o, o trabalho exp�e o problema do \textit{overhead} criado pelos sistemas de \textit{refresh} em mem�ria, o que pode degradar tanto o desempenho quanto o consumo energ�tico. Com o intuito de reduzir a necessidade de \textit{refresh} em mem�ria, o escalonamento de \textit{loops}, para melhor a aloca��o e acesso de dados na mem�ria se torna uma t�cnica promissora.

\section{Discuss�es}

	As STT-RAM vol�teis, dentro dos artigos estudados, mostraram-se aplic�veis como futuras substitutas para as SRAM. Isto se deve a diversos fatores os quais foram avaliados nos trabalhos, como o estudo da utiliza��o em v�rios n�veis de \textit{cache}, consumo energ�tico reduzido (tanto comparav�is com a STT-RAM n�o vol�til, quanto com as SRAM) e maior densidade para uma mesma �rea de mem�ria se comparada a SRAM.

	O principal impedimento para uso de STT-RAM a n�vel de mem�rias \textit{cache} � a alta lat�ncia de escrita. Com o intuito de mitigar este problema, utilizam-se t�cnicas como a redu��o da �rea da MTJ, aumento da temperatura de trabalho e principalmente a redu��o da corrente necess�ria para efetuar uma escrita em mem�ria. Observando os projetos de c�lulas STT-RAM avaliados nos trabalhos estudados, foi poss�vel verificar que existem \textit{tradeoffs} entre diversas caracter�sticas f�sicas. A Figura~\ref{fig:we} mostra a rela��o entre tempo de reten��o, lat�ncia de escrita, estabilidade, corrente de escrita e taxa de \textit{refresh}.

	 \begin{figure}[!h]
	 \centering
	 	\begin{tikzpicture}
	 	\tkzKiviatDiagram[scale=0.5,label distance=.5cm,
	 	    radial  = 5,
	 	    gap     = 3,  
	 	    lattice = 2]{Reten��o,Lat�ncia de escrita,Estabilidade,Corrente,\textit{Refresh}}
	 	\tkzKiviatLine[thick, color=blue, mark=none, fill=blue!20, opacity=.5](2,2,2,2,0)
	 	\tkzKiviatLine[thick, color=red, mark=none, fill=red!20,opacity=.5](1,1,1,1,2)    
	 	%\tkzKiviatGrad[prefix=,unity=100,suffix=\ �](1)  
	 	\end{tikzpicture}
	 	\caption{\textit{Tradeoffs} entre as caracater�sticas da c�lula STT-RAM.}
	 	\label{fig:we}
	 \end{figure}

	Tomando como base projetos completamente opostos de STT-RAM, o gr�fico demonstra que ao reduzir a �rea da MTJ, consegue-se uma redu��o na lat�ncia e corrente de escrita. Em contrapartida, h� a redu��o da reten��o dos dados e da estabilidade da MTJ, tornando-se assim necess�ria a utiliza��o de \textit{refresh}. J� com uma MTJ de maior �rea, obtemos maior reten��o dos dados e melhor estabilidade, sem a necessidade de \textit{refresh}. Contudo, aumenta-se a lat�ncia e corrente de escrita. A decis�o de utilizar um determinado tamanho de �rea para MTJ vai da necessidade de projeto empregado.

	Dentro do contexto das STT-RAM que necessitam de \textit{refresh} para manter a corretude dos dados, os artigos estudados utilizaram-se de t�cnicas auxiliaries tanto em \textit{hardware} como em \textit{software}. Os mecanismos avaliados em \textit{hardware} inclu�ram desde \textit{refreshes} estilo DRAM, hierarquias de \textit{caches} h�bridas (SRAM, STT-RAM tanto vol�til quanto n�o vol�til) e o uso de contadores que definem a necessidade ou n�o de fazer \textit{refresh}. J� em \textit{software}, uma das t�cnicas ordena os dados em tempo de compila��o para evitar \textit{refreshes} desnecess�rios. Estas utilizam heur�sticas para prever dados que ser�o acessados gra�as � localidade temporal. Outra possibilidade encontrada foi o escalonamento dos dados com a finalidade de explorar a localidade especial, o que tamb�m reduz a quantidade de \textit{refreshes}.

	Apesar das t�cnicas apresentadas acima proverem meios para o uso de STT-RAM vol�teis, e manter a premissa de baixo consumo energ�tico, um estudo apresentado em~\cite{kim2016} exibe problemas na utiliza��o de \textit{refreshes} estilo DRAM para manter os dados nestas mem�rias atualizados. Isto porque uma falha na reten��o de STT-RAM vol�til n�o pode ser evitada usando \textit{refresh} estilo DRAM devido � natureza estoc�stica desta falha. Al�m disto, existem tr�s tipos de falha que a STT-RAM vol�til pode apresentar, sendo que estas podem ocorrer nas opera��es de leitura, de escrita e na reten��o do dado. Sugeriu-se o uso de ECC e \textit{scrubbing} para o uso de STT-RAM vol�til, o qual devido ao \textit{overhead} gerado por estas t�cnicas, n�o obteve bons resultados tanto em desempenho quanto em consumo energ�tico.

\section{Conclus�es}

� v�lido o uso de STT-RAM como mem�rias vol�teis em sistemas de mem�ria?

\bibliographystyle{sbc}
\bibliography{survey-ham}

\end{document}