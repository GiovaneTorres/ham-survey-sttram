\documentclass[10pt, pdf,xcolor=pdftex,dvipsnames,table]{beamer}

\usepackage[brazil]{babel}
\usepackage[utf8] {inputenc}
\usepackage [T1] {fontenc}
\usepackage{pgfpages}
\usepackage{times}
\usepackage{amsmath,amssymb}
\usepackage{graphicx}
\usepackage{color}
\usepackage{hyperref}
\usepackage{pxfonts,txfonts}
\usepackage{url}
\usefonttheme{structurebold}
\usepackage{hyphenat}
\usepackage{multicol}
\usepackage{todonotes}
\usepackage{amssymb}
\usepackage{latexsym}
\usepackage{fixltx2e}

%\usepackage[usenames,dvipsnames]{xcolor}
\usepackage{tikz}
\usepackage{tkz-kiviat}
\usetikzlibrary{positioning}
\usetikzlibrary{arrows}

\usepackage{caption}
\usepackage{subcaption}


\newcommand{\x}{\textbf{\textcolor{DarkGreen}{$\surd$}}}
\newcommand{\xx}{\textbf{\textcolor{DarkBlue}{$\odot$}}}
\newcommand{\xxx}{\textbf{\textcolor{DarkRed}{$\times$}}}


\usetheme{Amsterdam}

\setbeamertemplate{navigation symbols}{}


\setbeameroption{hide notes}

%==================================================================================
\renewcommand{\evento}{Hierarquias Avançadas de Memória}
\title{Um estudo sobre o uso de memórias STT-RAM voláteis em diferentes níveis de memória cache}
\author[]{\textbf{Giovane de O. Torres\inst{1}, Rodrigo M. Duarte\inst{1}}\\
}

%%%%%%%%%%%%%%%%%%%%%%%%%%%%%%%%%%%%%%%%
% Instituição
%%%%%%%%%%%%%%%%%%%%%%%%%%%%%%%%%%%%%%%%

\institute{Pós-Graduação em Computação \\ Universidade Federal de Pelotas \\
\url{} 
}


%%%%%%%%%%%%%%%%%%%%%%%%%%%%%%%%%%%%%%%%
% Data
% sem data (\date{})
% dia de hoje  \date{\today}
%%%%%%%%%%%%%%%%%%%%%%%%%%%%%%%%%%%%%%%%
\date{Agosto de 2017}

\begin{document}

\frame{\titlepage}
\frame{\tableofcontents}

%%%%%%%%%%%%%%%%%%%%%%%%%%%%%%%%%%%%%%%%%%%%%
% Conteúdo da Apresentação
%%%%%%%%%%%%%%%%%%%%%%%%%%%%%%%%%%%%%%%%%%%%%

\section{Introdução}

\frame
{
\frametitle{Introdução}
		\begin{block}{Porque novos tipos de memória?}
			\begin{itemize}
				\item SRAMs consomem muita energia.
				\begin{itemize}
					\item Tanto estática como dinâmica.
				\end{itemize}
				\item Reduzir \textit{gap} entre processador e memória.
				\begin{itemize}
					\item Reduzir latência da memória para aumentar desempenho.
				\end{itemize}
			\end{itemize}	 
		\end{block}
		\begin{block}{Tecnologias Emergentes de Memória}
			\begin{itemize}
				\item Memórias não voláteis.
				\begin{itemize}
					\item Promissoras, porém possuem alta latência de escrita.
					\item STT-RAM Volátil.
				\end{itemize}
			\end{itemize}	 
		\end{block}
}

\section{STT-RAM e Volatilidade}

\frame
{
\frametitle{STT-RAM e Volatilidade}
		\begin{block}{STT-RAM}
			\begin{itemize}
				\item Alta latência de escrita e baixa de leitura.
				\item Baixo consumo energético (\textit{leakage} desprezível).
				\item Maior densidade de memória por área física.
			\end{itemize}	 
		\end{block}
		\begin{block}{Funcionamento}
			\begin{itemize}
				\item Possui duas camadas magnéticas intercaladas por um material isolante.
				\item Essa composição é conhecida como MTJ (Magnetic Tunnel Junction).
			\end{itemize}	 
		\end{block}
}

\frame
{
\frametitle{STT-RAM e Volatilidade}
\begin{tikzpicture}

\draw[black,fill=black!30] (0,0) rectangle (3,0.5);		% Left reference layer
\draw[black] (0,0.5) rectangle (3,0.8);					% Left barrier
\draw[black,fill=black!30] (0,0.8) rectangle (3,1.3);	% Left free layer

\draw[<-,very thick] (1,1.05) -- (2,1.05);				% Arrow of reference layer
\draw[<-,very thick] (1,0.25) -- (2,0.25);				% Arrow of free layer

\draw[black,fill=black!30] (8,0) rectangle (11,0.5);	% Right reference layer
\draw[black] (8,0.5) rectangle (11,0.8);				% Right barrier
\draw[black,fill=black!30] (8,0.8) rectangle (11,1.3);	% Right free layer

\draw[<-,very thick] (9,1.05) -- (10,1.05);				% Arrow of reference layer
\draw[->,very thick] (9,0.25) -- (10,0.25);				% Arrow of free layer

\node at (5.5,1.05) {Camada de referência}; 			% Text: reference layer
\node at (5.5,0.25) {Camada livre}; 					% Text: free layer

\draw[-] (1.5,0) -- (1.5,-0.5);							% Left downwards small line 
\draw[-] (9.5,0) -- (9.5,-0.5);							% Right downwards small line 

\draw[black] (1.5,-0.6) circle (0.1);					% Left downwards small circle
\draw[black] (9.5,-0.6) circle (0.1);					% Right downwards small circle

\draw[-] (1.5,1.3) -- (1.5,1.8);						% Left upwards small line 
\draw[-] (9.5,1.3) -- (9.5,1.8);						% Right upwards small line 

\draw[black] (1.5,1.9) circle (0.1);					% Left upwards small circle
\draw[black] (9.5,1.9) circle (0.1);					% Right upwards small circle

\node at (1.5,-1.2) {"1"~lógico};						% Text: logical 1
\node at (1.5,-1.7) {Alta resistividade};				% Text: high resistivity

\node at (9.5,-1.2) {"0"~lógico};						% Text: logical 0
\node at (9.5,-1.7) {Baixa resistividade};				% Text: low resistivity

\end{tikzpicture}	
}

\frame
{
\frametitle{STT-RAM e Volatilidade}
		\begin{block}{STT-RAM Volátil}
			\begin{itemize}
				\item Redução da área física da MTJ.
				\item Aumento da temperatura de trabalho.
				\item Redução da corrente de escrita.
			\end{itemize}	 
		\end{block}
		\begin{block}{Contrapartida}
			\begin{itemize}
				\item Redução do tempo de retenção do dado.
				\item Necessidade de \textit{refresh} para manter os dados corretamente armazenados.
			\end{itemize}	 
		\end{block}
}

\frame
{
\frametitle{STT-RAM e Volatilidade}
	\begin{figure}[!h]
	 \centering
	 	\begin{tikzpicture}
	 	\tkzKiviatDiagram[scale=0.4,label distance=.5cm,
	 	    radial  = 5,
	 	    gap     = 3,  
	 	    lattice = 2]{Retenção,Latência de escrita,Estabilidade,Corrente,\textit{Refresh}}
	 	\tkzKiviatLine[thick, color=blue, mark=none, fill=blue!20, opacity=.5](2,2,2,2,0)
	 	\tkzKiviatLine[thick, color=red, mark=none, fill=red!20,opacity=.5](1,1,1,1,2)    
	 	%\tkzKiviatGrad[prefix=,unity=100,suffix=\ £](1)  
	 	\end{tikzpicture}
	 	\caption{\textit{Tradeoffs} entre as caracaterísticas da célula STT-RAM.}
	 	\label{fig:we}
	 \end{figure}	
}

\section{Técnicas para uso da STT-RAM volátil em cache}

\frame
{
\frametitle{Técnicas para uso da STT-RAM volátil em cache}
		\begin{block}{Técnicas em Hardware}
			\begin{itemize}
				\item \textit{Refresh} simples estilo DRAM.
				\item Hierarquia de \textit{caches} Hibridas.
				\item Uso de contadores.
			\end{itemize}	 
		\end{block}
		\begin{block}{Técnicas em Software}
			\begin{itemize}
				\item Heurística para exploração de localidade temporal.
				\item Escalonamento para exploração localidade espacial.
			\end{itemize}
		\end{block}
}

\frame
{
\frametitle{Técnicas para uso da STT-RAM volátil em cache}
		\begin{block}{Resultados Gerais}
			\begin{itemize}
				\item Ganhos de desempenho se comparado a STT-RAM não volátil.
				\item Redução de consumo energético em relação a SRAM.
				\item Maior densidade de memória por área física.
			\end{itemize}	 
		\end{block}
}

\frame
{
\frametitle{Técnicas para uso da STT-RAM volátil em cache}
		\begin{block}{Ressalvas}
			\begin{itemize}
				\item Falha de retenção tem problema estocástico.
				\item Três tipos de falha em STT-RAM Volátil.
				\begin{itemize}
					\item Escrita: Dado pode não ser gravado corretamente.
					\item Leitura: Pode ocorrer leitura errada ou alteração indevida.
					\item Retenção: Não houve atualização durante o tempo de vida do dado.
				\end{itemize} 
			\end{itemize}	 
		\end{block}
		\begin{block}{Possíveis Soluções}
			\begin{itemize}
				\item Uso de \textit{Scrubbing} e ECC.
				\begin{itemize}
					\item Não se demostra eficiente, tanto em desempenho quanto em consumo energético.
				\end{itemize} 
			\end{itemize}	 
		\end{block}
}

\section{Conclusões}

\frame
{
\frametitle{Conclusões}
		\begin{block}{STT-RAM é promissora candidata para substituir SRAMs}
			\begin{itemize}
				\item Baixa latência e consumo energético.
				\item Alta densidade de memória.
			\end{itemize}	 
		\end{block}
		\begin{block}{Aplicação ainda precisa de maiores estudos}
			\begin{itemize}
				\item Falhas estocásticas.				
				\item Técnicas de \textit{refresh} podem não ser suficientes.
			\end{itemize}
		\end{block}
}


\titlepage
\end{document}

