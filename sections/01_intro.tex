\section{Introdu��o}
	\begin{itemize}
		\item Sistemas de mem�ria. (desempenho e energia).
		\item Meios para melhorar o desempenho (lat�ncia e energia).
		\item Novas tecnologias de mem�ria (foco STT-RAM).
	\end{itemize}

\subsection{NVM (STT-RAM)}
	\begin{enumerate}
		\item Uso de caches
		\item Mem�ria n�o vol�til (retem informa��o por tempo indefinido).
		\item Reduzir a caracter�stica de n�o-volatilidade (caches est�o sujeitas a grandes quantidades de escrita, sendo os dados alterados com o passar do tempo, n�o sendo necess�rio a reten��o por longos per�odos de tempo.)
		\item A ideia principal � reduzir a �rea planar da c�lula de mem�ria com o intuito de reduzir principalmente a lat�ncia de escrita. Como consequ�ncia reduz-se a capacidade de reten��o do dado por tempo indeterminado. Com isso consegue uma maior densidade de mem�ria no mesmo espa�o ocupado por uma SRAM tradicional. 
	\end{enumerate}

\subsection{Objetivos do trabalho}
      Realizar um estudo sobre trabalhos que abordem a �rea da STT-RAM vol�til, usada em diferentes n�veis de cache. Avaliando quais as melhores abordagens para a utiliza��o desta categoria de mem�ria.