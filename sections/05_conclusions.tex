\section{Conclus�es}

Este trabalho fez um estudo acerca do uso de STT-RAM vol�teis em diferentes n�veis de \textit{cache}. A ideia principal foi fazer uma an�lise de quais as t�cnicas empregadas para a otimiza��o o uso destas mem�rias. Para isto, diversos trabalhos relacionados foram estudados, os quais demonstraram diversos m�todos que variam do \textit{hardware} at� o \textit{software}. Os resultados apresentados demonstram que a utiliza��o de STT-RAM vol�til � promissora como substituta das SRAM, devido a dois principais fatores: Baixo consumo energ�tico e alta densidade de mem�ria. Contudo, estudos mais recentes apontaram que uma investiga��o mais aprofundada a respeito de STT-RAM vol�teis deve ser feita -- isto porque diversas falhas foram observadas nas opera��es de mem�ria, o que incluem escritas e leituras, al�m de que falhas na reten��o do dado da STT-RAM possuem natureza estoc�stica. Com isto, t�cnicas mais aprimoradas de \textit{refresh} tornam-se necess�rias para que a utiliza��o de STT-RAM vol�teis tornem se vi�veis.